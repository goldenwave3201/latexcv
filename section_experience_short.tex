%Section: Work Experience at the top
\sectionTitle{项目经历}{\faCode}
 
\begin{experiences}
			
 \experience
    {2017年9月}{PM2.5预测}{台湾大学}{Python编程}
    {}{
    	\begin{itemize}
    		\item Matlab和C混合编程实现图像的SIFT特征提取,并将其应用于目标检测和分类;
    		\item 通过矩阵分解实现基于隐因子模型的协同过滤,学习用户的喜好偏向,实现推荐系统。
    		\item Python编程实现基于求解Poissn方程的图像无缝拼接,以及基于热传导的图像去噪。
    		%\item \faGithub: 
			%    		\link{https://github.com/huajh/sift}{sift}, 
            %            \link{https://github.com/huajh/mf_re_sys}{MFResys},
            %            \link{https://github.com/huajh/Poisson_image_editing}{PoissonImageEditing},
            %            \link{https://https://github.com/huajh/Image_denoising}{ImageDenoising}
         \end{itemize}
       }
    {逻辑回归, Python}
  %\emptySeparator
  \experience
    { 2014年4月} {视频中的行为检测与识别方法研究}{浙江大学}{ 独立开发}
    {}    {
                      \begin{itemize}
                        \item 提取视频序列的\emph{时空特征点},并用K-Means对兴趣点\emph{聚类}并建立\emph{词库},得到词袋模型; 
                        \item 采用无监督的pLSA/LDA模型推断后验概率P(动作|词)实现动作类别归类;
                        \item 以及采用监督学习(KNN,SVM)对每帧图像的字典分类;                    
                        \item 提出一种简单的投票(``voting'')方法 实现单相机中的多目标检测任务。
                        \item \faGithub: \link{https://github.com/huajh/action_recognition} {github.com/huajh/action\_recognition}                                                                                          
                      \end{itemize}
                    }
                    {动作识别, 聚类, LDA, Voting, Bag of Words}
	
		
\end{experiences}
